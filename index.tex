% Options for packages loaded elsewhere
% Options for packages loaded elsewhere
\PassOptionsToPackage{unicode}{hyperref}
\PassOptionsToPackage{hyphens}{url}
\PassOptionsToPackage{dvipsnames,svgnames,x11names}{xcolor}
%
\documentclass[
  letterpaper,
  DIV=11,
  numbers=noendperiod]{scrreprt}
\usepackage{xcolor}
\usepackage{amsmath,amssymb}
\setcounter{secnumdepth}{5}
\usepackage{iftex}
\ifPDFTeX
  \usepackage[T1]{fontenc}
  \usepackage[utf8]{inputenc}
  \usepackage{textcomp} % provide euro and other symbols
\else % if luatex or xetex
  \usepackage{unicode-math} % this also loads fontspec
  \defaultfontfeatures{Scale=MatchLowercase}
  \defaultfontfeatures[\rmfamily]{Ligatures=TeX,Scale=1}
\fi
\usepackage{lmodern}
\ifPDFTeX\else
  % xetex/luatex font selection
\fi
% Use upquote if available, for straight quotes in verbatim environments
\IfFileExists{upquote.sty}{\usepackage{upquote}}{}
\IfFileExists{microtype.sty}{% use microtype if available
  \usepackage[]{microtype}
  \UseMicrotypeSet[protrusion]{basicmath} % disable protrusion for tt fonts
}{}
\makeatletter
\@ifundefined{KOMAClassName}{% if non-KOMA class
  \IfFileExists{parskip.sty}{%
    \usepackage{parskip}
  }{% else
    \setlength{\parindent}{0pt}
    \setlength{\parskip}{6pt plus 2pt minus 1pt}}
}{% if KOMA class
  \KOMAoptions{parskip=half}}
\makeatother
% Make \paragraph and \subparagraph free-standing
\makeatletter
\ifx\paragraph\undefined\else
  \let\oldparagraph\paragraph
  \renewcommand{\paragraph}{
    \@ifstar
      \xxxParagraphStar
      \xxxParagraphNoStar
  }
  \newcommand{\xxxParagraphStar}[1]{\oldparagraph*{#1}\mbox{}}
  \newcommand{\xxxParagraphNoStar}[1]{\oldparagraph{#1}\mbox{}}
\fi
\ifx\subparagraph\undefined\else
  \let\oldsubparagraph\subparagraph
  \renewcommand{\subparagraph}{
    \@ifstar
      \xxxSubParagraphStar
      \xxxSubParagraphNoStar
  }
  \newcommand{\xxxSubParagraphStar}[1]{\oldsubparagraph*{#1}\mbox{}}
  \newcommand{\xxxSubParagraphNoStar}[1]{\oldsubparagraph{#1}\mbox{}}
\fi
\makeatother


\usepackage{longtable,booktabs,array}
\usepackage{calc} % for calculating minipage widths
% Correct order of tables after \paragraph or \subparagraph
\usepackage{etoolbox}
\makeatletter
\patchcmd\longtable{\par}{\if@noskipsec\mbox{}\fi\par}{}{}
\makeatother
% Allow footnotes in longtable head/foot
\IfFileExists{footnotehyper.sty}{\usepackage{footnotehyper}}{\usepackage{footnote}}
\makesavenoteenv{longtable}
\usepackage{graphicx}
\makeatletter
\newsavebox\pandoc@box
\newcommand*\pandocbounded[1]{% scales image to fit in text height/width
  \sbox\pandoc@box{#1}%
  \Gscale@div\@tempa{\textheight}{\dimexpr\ht\pandoc@box+\dp\pandoc@box\relax}%
  \Gscale@div\@tempb{\linewidth}{\wd\pandoc@box}%
  \ifdim\@tempb\p@<\@tempa\p@\let\@tempa\@tempb\fi% select the smaller of both
  \ifdim\@tempa\p@<\p@\scalebox{\@tempa}{\usebox\pandoc@box}%
  \else\usebox{\pandoc@box}%
  \fi%
}
% Set default figure placement to htbp
\def\fps@figure{htbp}
\makeatother





\setlength{\emergencystretch}{3em} % prevent overfull lines

\providecommand{\tightlist}{%
  \setlength{\itemsep}{0pt}\setlength{\parskip}{0pt}}



 


\KOMAoption{captions}{tableheading}
\makeatletter
\@ifpackageloaded{bookmark}{}{\usepackage{bookmark}}
\makeatother
\makeatletter
\@ifpackageloaded{caption}{}{\usepackage{caption}}
\AtBeginDocument{%
\ifdefined\contentsname
  \renewcommand*\contentsname{Table of contents}
\else
  \newcommand\contentsname{Table of contents}
\fi
\ifdefined\listfigurename
  \renewcommand*\listfigurename{List of Figures}
\else
  \newcommand\listfigurename{List of Figures}
\fi
\ifdefined\listtablename
  \renewcommand*\listtablename{List of Tables}
\else
  \newcommand\listtablename{List of Tables}
\fi
\ifdefined\figurename
  \renewcommand*\figurename{Figure}
\else
  \newcommand\figurename{Figure}
\fi
\ifdefined\tablename
  \renewcommand*\tablename{Table}
\else
  \newcommand\tablename{Table}
\fi
}
\@ifpackageloaded{float}{}{\usepackage{float}}
\floatstyle{ruled}
\@ifundefined{c@chapter}{\newfloat{codelisting}{h}{lop}}{\newfloat{codelisting}{h}{lop}[chapter]}
\floatname{codelisting}{Listing}
\newcommand*\listoflistings{\listof{codelisting}{List of Listings}}
\makeatother
\makeatletter
\makeatother
\makeatletter
\@ifpackageloaded{caption}{}{\usepackage{caption}}
\@ifpackageloaded{subcaption}{}{\usepackage{subcaption}}
\makeatother
\usepackage{bookmark}
\IfFileExists{xurl.sty}{\usepackage{xurl}}{} % add URL line breaks if available
\urlstyle{same}
\hypersetup{
  pdftitle={Naufalziyadh Alif Bintang Shaeky},
  pdfauthor={18224077 Naufalziyadh Alif Bintang S},
  colorlinks=true,
  linkcolor={blue},
  filecolor={Maroon},
  citecolor={Blue},
  urlcolor={Blue},
  pdfcreator={LaTeX via pandoc}}


\title{Naufalziyadh Alif Bintang Shaeky}
\usepackage{etoolbox}
\makeatletter
\providecommand{\subtitle}[1]{% add subtitle to \maketitle
  \apptocmd{\@title}{\par {\large #1 \par}}{}{}
}
\makeatother
\subtitle{Portfolio Asesmen II-2100 KIPP}
\author{18224077 Naufalziyadh Alif Bintang S}
\date{2025-09-15}
\begin{document}
\maketitle

\renewcommand*\contentsname{Table of contents}
{
\hypersetup{linkcolor=}
\setcounter{tocdepth}{2}
\tableofcontents
}

\bookmarksetup{startatroot}

\chapter{}\label{section}

Halo semua!

Saya Naufalziyadh Alif Bintang S, mahasiswa aktif dari Sekolah Teknik
Elektro dan Informatika (STEI) ITB, mengambil program studi Sistem dan
Teknologi Informasi (STI). Saat ini saya fokus mendalami perpaduan
antara teknologi, sistem manajemen, dan strategi bisnis.

Mari terhubung dan berdiskusi! Anda bisa menemukan saya di Instagram:
(\textbf{naufalziyadh.abs?}).

\bookmarksetup{startatroot}

\chapter{UTS-1 All About Me}\label{uts-1-all-about-me}

Naufalziyadh Alif Bintang Shaeky (Falzi) 18224077 --- Sistem dan
Teknologi Informasi Halo semuanya, aku Falzi. Katanya sih aku ini ESFP,
si penghibur yang energinya kayak gak habis-habis. Buat aku, hidup tuh
kayak panggung besar, dan tiap hari selalu ada aja hal baru yang bisa
dibikin seru. Dunia ini gak cuma tentang masalah, tapi tentang gimana
kita nikmatin prosesnya bareng orang-orang di sekitar kita. Kuliah di
STI ngajarin aku soal logika dan sistem, tapi jujur, tempat ternyaman
aku justru di tengah keramaian. Aku paling senang ikut kegiatan, ngobrol
sama orang baru, dan ngerasain energi dari suasana yang rame. Buatku,
hal paling keren dari sebuah sistem adalah interaksi antar manusia.
Soalnya, suasana asik itu muncul kalau semua orang bisa nyatu dan
ngerasa senang bareng. Aku juga termasuk orang yang peka sama suasana.
Kalau lagi kumpul terus suasananya mulai kaku, aku sering jadi yang
pertama sadar. Biasanya langsung mikir, ``kok sepi sih?'' Terus refleks
nyari cara biar semua orang bisa ketawa lagi. Kadang cukup lewat obrolan
kecil atau dengerin temen curhat. Yang penting semua orang ngerasa
nyaman. Beberapa waktu terakhir aku sempat ngalamin beberapa hal yang
gak enak, termasuk soal akademik. Tapi aku belajar buat gak terus
tenggelam di situ. Aku lihat itu sebagai kesempatan buat tumbuh dan
belajar hal baru. Kalau satu jalan ketutup, ya cari cara lain. Kadang
malah ketemu hal yang lebih seru dari yang direncanain. Aku gak selalu
semangat setiap waktu, kadang juga turun. Tapi justru dari situ aku
belajar buat bangun hubungan yang kuat sama orang lain. Hubungan yang
gak cuma karena seru di awal, tapi juga karena kita saling dukung dan
peduli. Mimpiku sederhana. Aku pengen punya lingkungan pertemanan yang
jadi tempatnya belajar bareng, saling peduli, dan bebas nyalurin ide.
Aku pengen dikelilingi orang-orang yang bukan cuma pintar, tapi juga
punya hati yang hangat. Kalau boleh dibilang, aku nyari temen yang jago
mikir tapi juga baik hati. Temen yang bisa ngobrol dari hal receh sampai
hal berat, tapi tetap bikin semua orang ngerasa diterima. Tempat di mana
kita ngerasa punya rumah bareng.

Itu aku, Falzi. Anak ESFP yang suka bikin suasana rame, tapi juga senang
denger cerita orang. Buat aku, hidup itu soal berbagi energi positif dan
bikin setiap momen jadi sesuatu yang bisa diinget bareng-bareng. Salam
kenal ya, halo semua!

\bookmarksetup{startatroot}

\chapter{UTS 2 - My Song For You}\label{uts-2---my-song-for-you}

(Verse 1) Hei, kamu! Ngapain termangu? Mata menatap jauh ke sepatu
Mikirin omongan si itu, si ini Takut salah, takut nggak keren lagi

Dunia berputar cepat sekali Tik-tok, nggak akan menanti Kamu masih di
situ, sibuk meragu Kapan mau maju?

(Pre-Chorus) Jangan cuma jadi penonton Yang tepuk tangan sambil gigit
jempol Hidup ini bukan film orang Ini panggungmu, ayo dong!

(Chorus) Hidup ini cuma sekali! (Hei!) Coba semua yang kau ingini! Jatuh
bangun itu pasti! Jangan sampai menyesal nanti! Aw aw aw awww Mendingan
gagal daripada tak nyali! Aw aw aw awww Gas aja dulu, pikir belakangan!

(Verse 2) Mau cat rambut warna pelangi? Mau teriak di puncak tertinggi?
Mau belajar dansa atau main biola? Kenapa harus nunggu tua?

Standar mereka bikin gila Nggak ada habisnya, percuma Ini badanmu, ini
suaramu Terserah kamu, mau dibawa ke mana!

(Pre-Chorus) Jangan cuma jadi penonton Yang tepuk tangan sambil gigit
jempol Hidup ini bukan film orang Ini panggungmu, ayo dong!

(Chorus) Hidup ini cuma sekali! (Hei!) Coba semua yang kau ingini! Jatuh
bangun itu pasti! Jangan sampai menyesal nanti! Aw aw aw awww Mendingan
gagal daripada tak nyali! Aw aw aw awww Gas aja dulu, pikir belakangan!

(Bridge) Mungkin nanti kamu lebam Mungkin ceritanya kelam Tapi itu jauh
lebih baik Daripada buku kosong tanpa coretan!

Aku cuma mau bilang\ldots{} Dengerin aku\ldots{} SAYANGGGGGGGG!!! Ayo
bangun!

(Chorus) Hidup ini cuma sekali! (Hei!) Coba semua yang kau ingini! Jatuh
bangun itu pasti! Jangan sampai menyesal nanti! Aw aw aw awww Mendingan
gagal daripada tak nyali! Aw aw aw awww Gas aja dulu, pikir belakangan!

(Outro) Coba semuanya! (Jangan menyesal!) Hidup sekali! (Coba semuanya!)
Aw aw aw awww! Ya! Jangan sampai menyesal! (Musik berhenti mendadak di
ketukan terakhir)

\bookmarksetup{startatroot}

\chapter{UTS-3 My Stories for You}\label{uts-3-my-stories-for-you}

My Story For You Halo, ini aku lagi belajar buat UTS Jarkom, tapi
tiba-tiba keinget kalau ada tugas bikin cerita. Jadi yaudah, cerita dulu
deh sebelum otakku keburu meledak gara-gara subnetting dan protokol yang
entah kenapa nggak masuk-masuk ke kepala. Kadang, di tengah tumpukan
catatan dan angka-angka aneh itu, aku suka kepikiran hal-hal yang justru
nggak ada hubungannya sama pelajaran. Kayak hari ini, aku kepikiran
semua perjalanan yang udah kulewatin sejak pertama kali masuk ITB.
Mungkin ini waktu yang pas buat berhenti sebentar, tarik napas, dan
ngeliat ke belakang, bukan buat nyesel, tapi buat nginget gimana
semuanya dimulai. Kalau ditarik mundur ke akhir tahun 2023, hidupku
waktu itu cuma seputar tryout dan pengumuman. Masa-masa SNBP adalah fase
di mana aku ngerasa semuanya bisa dikontrol. Aku yakin banget bisa masuk
universitas impian tanpa drama. Tapi ternyata, dunia punya cara unik
buat bikin kita sadar bahwa nggak semua bisa sesuai rencana. Tryout
pertamaku cuma dapat nilai lima ratusan, angka yang awalnya bikin aku
bengong lama di depan layar. Dari situ aku baru ngerti, percaya diri
tanpa usaha itu sama aja kayak jalan di kabut, keliatannya aman, padahal
nyasar pelan-pelan. Tapi nilai kecil itu justru jadi titik balik. Aku
mulai rajin belajar, ikut bimbingan, dan pelan-pelan mulai ngerasa bahwa
mungkin kegagalan itu bukan akhir, tapi peringatan kecil supaya aku
nggak main-main. Hari pengumuman SNBP datang, dan seperti dugaan buruk
yang berulang di kepala, aku gagal. Aku nggak diterima. Rasanya waktu
itu kayak seluruh usaha selama berbulan-bulan tiba-tiba nggak ada
artinya. Tapi anehnya, setelah nangis bentar dan diem lama, aku malah
ngerasa tenang. Aku bilang ke diri sendiri, mungkin Tuhan cuma nyuruh
aku muter dikit sebelum nyampe. Lalu aku ikut SNBT. Hasilnya? Nyaris.
Tinggal satu langkah lagi, tapi kursi itu bukan buatku. Kalau dipikir
sekarang, mungkin itu cara semesta bilang, ``Belum, tapi nanti.'' Jadi
aku lanjut ambil ujian mandiri ITB. Aku belajar siang malam, bahkan
sehari setelah wisuda masih buka catatan. Aku inget banget, ustaz di
pondok ngasih doa panjang banget sebelum aku berangkat. Dan ternyata,
doa itu beneran nyampe. Aku diterima di STEI-K ITB. Waktu baca
pengumuman itu, aku cuma bisa bengong. Dunia yang sempet terasa gelap
mendadak terang, dan aku nangis, bukan karena sedih, tapi karena
akhirnya aku bisa lega. Masuk ITB, rasanya kayak dilempar ke dunia baru
yang seru sekaligus menakutkan. Hari pertama kuliah, kampus lain sibuk
ospek, tapi kami malah dikasih mata kuliah Pancasila. Lucu banget,
pikirku waktu itu. Aku duduk di kelas, nggak kenal siapa-siapa, bingung
mau mulai dari mana. Tapi dari kelas itulah aku pertama kali ngerasa
punya tempat. Aku sekelompok sama orang-orang yang ternyata asik banget.
Dari kerja kelompok, ngobrolin hal random, sampai akhirnya sering
nongkrong bareng. Mereka yang awalnya cuma nama di daftar hadir,
lama-lama jadi orang yang selalu aku cari tiap kali ada kelas baru. Di
situlah aku sadar, ternyata pertemanan bisa tumbuh dari hal sekecil
``ayo kerjain tugas bareng, yuk.'' Lalu datang OSKM. Ah, ini bagian yang
susah banget dilupain. Capek, rame, berisik, tapi jujur, aku kangen.
Hari-harinya penuh tawa, teriakan, dan sedikit drama. Aku ketemu banyak
orang baru, bahkan sempet suka sama seseorang meski cuma sebentar. Tapi
justru dari situ aku belajar hal yang jauh lebih penting. OSKM ngajarin
aku cara ngeliat orang bukan cuma dari luar. Di antara semua keseruan
dan kehebohan itu, aku mulai ngerti gimana tiap orang punya ceritanya
sendiri. Ada yang berjuang, ada yang sembunyi di balik senyum, ada juga
yang cuma pengen didengar. Dan di tengah kerumunan itu, aku ngerasa
hidup. Setelah OSKM, hidup di kampus mulai berjalan seperti biasa.
Sampai akhirnya aku ikut kepanitiaan pertamaku: Sekolah Tour. Awalnya
cuma karena temen ngajak, aku mikirnya bakal seru aja kalau rame-rame.
Tapi ternyata dari keputusan se-random itu, banyak banget hal yang
berubah. Di awal, aku sempat nggak aktif beberapa hari karena sibuk sama
urusan lain. Waktu balik lagi, suasananya udah beda. Semua orang udah
akrab, udah punya ritme sendiri, dan aku merasa asing di antara tawa
mereka. Tapi orang-orang di sana ternyata baik banget. Mereka tetap
nyambut aku kayak nggak pernah ketinggalan apa-apa. Pelan-pelan aku
mulai nyaman lagi. Hari itu, kami keliling lab rumpun elektronika.
Mentor kami luar biasa enerjik. Suaranya lantang, matanya nyala waktu
ngejelasin setiap alat di meja. Aku cuma bisa manggut-manggut, berusaha
kelihatan paham padahal dalam hati cuma, ``Ini apaan sih?'' Tapi di
balik semua itu, suasananya hangat banget. Kami foto bareng, ketawa
bareng, dan bahkan sempat bercanda sampai lupa waktu. Tapi ada satu
momen di mana aku tiba-tiba merasa diam. Di tengah tawa orang-orang, aku
ngerasa jauh, kayak cuma penonton di cerita yang harusnya aku mainin
sendiri. Mungkin itu perasaan capek, tapi entah kenapa, bagian itu
justru paling kuingat sampai sekarang. Begitu acara selesai, aku pulang
ke Bandung. Rumahku kecil, tapi selalu terasa besar setiap kali aku
masuk. Bunda lagi masak di dapur, wangi tumisan langsung nyambut dari
pintu. ``Gimana hari ini?'' tanya Bunda. Aku cuma bisa nyengir, ``Capek
banget, Bun. Rasanya pengen tidur seharian.'' Tapi Bunda cuma ketawa
kecil dan bilang, ``Capek itu tandanya kamu lagi ngelakuin sesuatu yang
berharga.'' Aku diem lama waktu itu. Kalimat sederhana yang entah kenapa
terasa dalam banget. Mungkin itu alasan kenapa setiap kali aku ngerasa
lelah, aku selalu inget perkataan itu. Setelah hari itu, semuanya
berjalan cepat. UAS datang, tugas menumpuk, dan kepanitiaan mulai terasa
jauh. Grup panitia di HP cuma kubaca sekilas, niat bales tapi
ujung-ujungnya malah aku tutup lagi. Kadang aku pengen aktif lagi, tapi
selalu ada suara kecil di kepala yang bilang, ``Udah telat.'' Sampai
suatu malam, ada pesan masuk dari salah satu panitia. Pesannya pendek
aja, nanyain apakah aku masih mau ikut kegiatan lanjutan. Aku sempat
ragu mau jawab apa. Tapi kalimat terakhirnya bikin aku diem lama,
``Kalau gitu kamu ikut UAS-nya kan, kann?'' Entah kenapa, kalimat
sesederhana itu bikin dadaku anget. Ada orang yang masih inget aku,
bahkan ketika aku sendiri udah mulai lupa. Dan dari situ aku ngerti,
kadang yang kita butuhin cuma satu orang yang ngulurin tangan duluan.
Itu malam aku mikir lama. Tentang betapa seringnya aku mundur cuma
karena takut ketinggalan. Padahal sebenarnya nggak ada yang benar-benar
terlambat kalau kita mau balik lagi. Mungkin makna ``ikatan'' bukan cuma
soal siapa yang selalu bareng, tapi tentang siapa yang masih inget kita
ketika kita nggak ada. Sekolah Tour akhirnya bukan cuma kegiatan kampus
buatku, tapi titik di mana aku belajar tentang hubungan. Tentang rasa
lelah yang ternyata nggak sia-sia, tentang teman yang nggak pernah
berhenti percaya, dan tentang diriku sendiri yang belajar buat nggak
takut mulai dari awal. Sekarang, di tengah malam yang dingin dan
tumpukan catatan Jarkom yang belum aku sentuh lagi, aku senyum sendiri.
Dari anak SMA yang dulu gagal tryout, sekarang aku duduk di kampus
impian, nulis cerita tentang perjalanan yang nggak sempurna tapi nyata.
Semua perjuangan, tawa, bahkan rasa ragu, ternyata punya tempatnya
masing-masing. Mungkin nanti kalau aku baca tulisan ini lagi, aku bakal
bilang ke diri sendiri, ``Lihat, kamu dulu cuma pengen lulus tryout,
tapi sekarang kamu lagi nulis cerita hidupmu sendiri.'' Dan mungkin di
situlah inti dari semuanya. Bukan soal seberapa cepat sampai, tapi
seberapa tulus kita jalanin setiap langkahnya, bareng orang-orang yang
selalu ngingetin kita buat terus maju, meski pelan.

\bookmarksetup{startatroot}

\chapter{UTS-4 My SHAPE (Spiritual Gifts, Heart, Abilities, Personality,
Experiences)}\label{uts-4-my-shape-spiritual-gifts-heart-abilities-personality-experiences}

My SHAPE Halo, aku Falzi. Kali ini aku mau cerita tentang satu hal yang
lumayan menarik buat aku, yaitu My SHAPE. Buatku, ini bukan sekadar
tugas refleksi, tapi cara buat lebih kenal sama diri sendiri. Aku pengen
jujur dan apa adanya, karena setiap orang punya caranya sendiri buat
tumbuh dan bersinar. S --- Signature Strengths (Kekuatan Utamaku) Kalau
ditanya apa kekuatanku, aku nggak langsung mikir soal kemampuan
akademik. Aku lebih ngerasa kekuatanku ada di energi positif dan caraku
berinteraksi sama orang lain. Aku paling hidup kalau bisa ngobrol, kerja
bareng, atau bikin suasana jadi lebih santai. Dari situ biasanya muncul
ide-ide spontan yang nggak kepikiran sebelumnya. Selain itu, aku cepat
banget menyesuaikan diri. Di tempat baru, aku nggak butuh waktu lama
buat nyatu sama lingkungan. Aku senang denger cerita orang, jadi aku
mudah nyambung dengan berbagai karakter. Rasanya kayak punya tombol
``connect'' yang langsung aktif setiap kali ketemu orang baru. H ---
Heart (Hal yang Bikin Aku Peduli dan Bersemangat) Bagian ini yang paling
ngena buat aku. Hatiku paling hidup kalau lagi di tengah-tengah orang
yang punya semangat yang sama. Aku suka suasana ramai yang hangat, di
mana semua orang bisa jadi diri sendiri. Dari situ aku ngerasa, ternyata
kebahagiaan itu bukan soal pencapaian pribadi, tapi soal kebersamaan.
Aku punya nilai yang selalu aku pegang, yaitu membuat lingkungan di
sekitarku jadi lebih menyenangkan. Aku suka jadi bagian dari sesuatu
yang bisa bikin orang lain ngerasa diterima dan dihargai. Karena aku
percaya, kebahagiaan itu menular. A --- Aptitudes \& Acquired Skills
(Bakat dan Kemampuan yang Aku Kembangin) Kalau dibayangin kayak
perjalanan game, aku punya dua jenis kemampuan. Pertama, kemampuan alami
yang udah ada dari dulu, dan kedua, kemampuan yang aku bangun selama
proses belajar di kampus. Kemampuan alami yang paling aku rasain adalah
mudah bergaul dan punya empati. Aku bisa ngerasain perubahan suasana dan
tahu kapan harus ngomong, kapan harus dengerin. Ini bantu banget waktu
kerja kelompok atau ngurus kegiatan bareng teman-teman. Kemampuan yang
aku kembangin sekarang lebih ke arah analisis dan pemecahan masalah. Di
STI, aku belajar cara berpikir sistematis dan mencari solusi yang
efisien. Tapi aku juga pengen kemampuan itu tetap punya sentuhan
manusiawi, supaya hasilnya nggak cuma berguna secara teknis tapi juga
bisa membantu kehidupan orang lain. Selain itu, aku juga terus ngasah
kemampuan bicara di depan umum, karena aku suka banget berbagi energi
positif ke orang banyak. P --- Personality (Kepribadian yang Ngebentuk
Cara Aku Melihat Dunia) Aku termasuk orang yang spontan, terbuka, dan
penuh rasa ingin tahu. Aku lebih suka langsung turun ke lapangan
daripada cuma mikirin teori. Hal baru selalu bikin aku semangat, apalagi
kalau bisa ngerasain langsung pengalaman itu. Aku juga tipe orang yang
fokus ke momen sekarang. Aku percaya hidup itu lebih bermakna kalau
dijalani sepenuh hati, bukan cuma dipikirin. Kadang aku bisa terlalu
terbawa suasana, tapi dari situ juga aku belajar banyak hal tentang
keseimbangan antara perasaan dan logika. E --- Experiences (Pengalaman
yang Ngebentuk Aku Sampai Sekarang) Setiap pengalaman yang aku lewatin
punya ceritanya sendiri. Ada yang lucu, ada yang bikin jatuh, tapi
semuanya berharga. Aku pernah ngalamin masa-masa di mana semangatku
turun karena hasil yang nggak sesuai harapan. Tapi dari situ aku belajar
buat nggak berhenti. Aku sadar kalau tumbuh itu nggak selalu terasa
enak, tapi justru dari situ kita tahu sejauh apa kita udah berjalan.
Setiap kegagalan ngasih pelajaran tentang sabar dan percaya proses. Dan
setiap pertemuan dengan orang baru selalu ngasih sudut pandang yang
bikin aku berkembang. Semua itu ngebentuk aku jadi orang yang lebih kuat
dan lebih peduli. Itu versi SHAPE-ku. Buatku, ini bukan cuma kumpulan
poin tentang diri sendiri, tapi juga perjalanan buat mengenal siapa aku
sebenarnya. Aku belajar bahwa kekuatan itu bukan selalu soal kemampuan
besar, tapi soal keinginan buat terus mencoba, peduli, dan belajar dari
setiap momen. Hidup mungkin bisa dibilang kayak panggung, tapi aku nggak
pengen cuma tampil di atasnya. Aku pengen bikin panggung itu jadi tempat
di mana semua orang bisa bersinar bareng, saling dukung, dan tumbuh
sama-sama. Karena di akhir hari, yang paling penting bukan seberapa
besar cahayamu, tapi seberapa banyak cahaya yang bisa kamu bagikan ke
sekitar.

\bookmarksetup{startatroot}

\chapter{UTS 5 - My Personal Reviews}\label{uts-5---my-personal-reviews}

Link File
Penilaian:.\url{https://docs.google.com/spreadsheets/d/1aLlqpvPccQlsjSFBLzTL9BblCH6NPUJ0/edit?usp=sharing&ouid=109436108499234689094&rtpof=true&sd=true}\\
\strut \\
Hasil Self-Assessment UTS (URL: ii-2100.github.io/all-about-me/falzi)
6.1 Identifikasi Nama \& NIM penulis: Naufalziyadh Alif Bintang Shaeky
-- 18224077 (Sistem dan Teknologi Informasi) Penilai: Self-assessment
(Naufalziyadh Alif Bintang Shaeky) Catatan cakupan: Halaman beranda
menampilkan ``All About Me'' (UTS-1). Navigasi ke ``My Songs for You'',
``My Stories for You'', dan ``My SHAPE'' tersedia, masing-masing berisi
karya lengkap (teks, refleksi, dan audio untuk lagu). (II 2100) 6.2
Tinjauan Umum\\
\strut \\
● UTS-1 (All About Me): Tulisan memperkenalkan identitas dan kepribadian
penulis dengan gaya naratif reflektif yang kuat. Cerita disusun secara
personal dan jujur, dengan keseimbangan antara humor ringan dan refleksi
mendalam.\\
\strut \\
● UTS-2 (My Songs for You): Lirik lagu ``Terjang Batas'' menampilkan
pesan motivatif dengan bahasa yang santai dan penuh energi. Struktur
lirik tersusun rapi (verse--chorus--bridge), dengan pesan eksplisit
tentang keberanian mencoba.\\
\strut \\
● UTS-3 (My Stories for You): Cerita ``My Story for You'' memadukan
pengalaman akademik, sosial, dan personal secara padu. Narasinya hidup,
reflektif, dan memiliki kedalaman emosional yang kuat.\\
\strut \\
● UTS-4 (My SHAPE): Refleksi diri ditulis dengan sistematis dan
menggambarkan integrasi yang baik antara kekuatan pribadi, nilai, dan
pengalaman hidup. Setiap poin SHAPE saling terhubung dan disusun dengan
alur reflektif yang konsisten. 6.3 Tinjauan Spesifik + Skor (1--5)
6.3.1\\
\strut \\
UTS-1 --- All About Me Skor per kriteria: Orisinalitas 5, Keterlibatan
5, Humor 4, Wawasan 5 → Total 19/20 (95\%) Alasan singkat: Tulisan
personal, jujur, dan menggugah. Narasi lancar, dengan humor ringan dan
insight yang bermakna. Menunjukkan pemahaman diri dan kemampuan refleksi
yang matang. Saran perbaikan: Tambahkan satu-dua kalimat yang merangkum
``pesan utama'' di akhir untuk memperkuat penutupan reflektif. 6.3.2\\
\strut \\
UTS-2 --- My Songs for You Skor per kriteria: Orisinalitas 5,
Keterlibatan 5, Humor 4, Inspirasi 5 → Total 19/20 (95\%) Alasan
singkat: Lirik orisinal, energik, dan penuh semangat. Penggunaan bahasa
sehari-hari terasa hidup dan membangun koneksi dengan pendengar.
Struktur musikal kuat dan berirama. Saran perbaikan: Tambahkan sedikit
deskripsi tentang proses kreatif atau inspirasi di balik lagu agar
konteks emosinya lebih terasa. 6.3.3\\
\strut \\
UTS-3 --- My Stories for You Skor per kriteria: Orisinalitas 5,
Keterlibatan 5, Pengembangan Narasi 5, Inspirasi 5 → Total 20/20 (100\%)
Alasan singkat: Narasi personal yang hangat dan reflektif. Struktur
cerita rapi dan progresif, dengan detail emosional yang kuat. Penutupnya
memberi pesan moral yang jelas tanpa terasa menggurui. Saran perbaikan:
Tambahkan satu kalimat ringkasan di awal yang memperkenalkan pesan utama
agar pembaca langsung memahami arah narasi. 6.3.4\\
\strut \\
UTS-4 --- My SHAPE Skor per kriteria: Orisinalitas 5, Keterlibatan 5,
Pengembangan Narasi 5, Inspirasi 5 → Total 20/20 (100\%) Alasan singkat:
Refleksi diri sangat kuat, dengan gaya bahasa yang ringan dan jujur.
Setiap bagian SHAPE saling melengkapi dan menunjukkan pemahaman mendalam
tentang karakter serta nilai pribadi. Saran perbaikan: Pertimbangkan
menambahkan satu paragraf pembuka yang menjelaskan makna SHAPE bagi kamu
secara keseluruhan sebelum masuk ke subbagian.\\
\strut \\
6.4 Rekapitulasi Akhir N o 1 UTS Aspek Dinilai All About Me
Orisinalitas, Keterlibatan, Humor, Wawasan 2 3 4 My Songs for You My
Stories for You Orisinalitas, Keterlibatan, Humor, Inspirasi
Orisinalitas, Keterlibatan, Pengembangan, Inspirasi My SHAPE
Orisinalitas, Keterlibatan, Pengembangan, Inspirasi\\
\strut \\
Skor Total: 19/20 19/20 20/20 20/20\\
Persentase: 95\% 95\% 100\% 100\%\\
Keterangan: Sangat Baik Sangat Baik Sangat Baik Sangat Baik\\
Komentar Umum: Portofolio menunjukkan refleksi diri yang matang,
ekspresif, dan konsisten dalam gaya naratif. Semua karya menghadirkan
kepribadian yang jujur dan positif. Gaya tulisan komunikatif serta ritme
narasi yang hidup menjadi kekuatan utama.

\bookmarksetup{startatroot}

\chapter{UAS-1 My Concepts}\label{uas-1-my-concepts}

\section{\texorpdfstring{\textbf{Identitas Naratif: Rekayasa Makna dalam
Era Komunikasi
Digital}}{Identitas Naratif: Rekayasa Makna dalam Era Komunikasi Digital}}\label{identitas-naratif-rekayasa-makna-dalam-era-komunikasi-digital}

\begin{figure}[H]

{\centering \pandocbounded{\includegraphics[keepaspectratio]{My_Concepts/../images/identity.png}}

}

\caption{Identity}

\end{figure}%

Konsep saya berangkat dari satu kegelisahan mendasar: di tengah derasnya
arus komunikasi digital, manusia semakin fasih berbicara, tetapi semakin
jarang \textbf{benar-benar menyampaikan siapa dirinya}. Kita hidup dalam
dunia yang penuh pesan, namun miskin makna. Fenomena ini bukan sekadar
krisis komunikasi, melainkan \textbf{krisis identitas dalam komunikasi}.

Saya menyebut fondasi untuk menjawab krisis ini sebagai
\textbf{Identitas Naratif}---sebuah kerangka konseptual yang memandang
manusia sebagai makhluk pencerita yang membangun makna hidup melalui
rangkaian pengalaman, refleksi, dan tujuan.

Identitas naratif bukan sekadar cerita tentang masa lalu, melainkan
\textbf{arsitektur makna} yang menghubungkan: - pengalaman yang
membentuk nilai, - kesadaran diri di masa kini, - dan arah hidup yang
ingin dituju.

Dalam konteks komunikasi interpersonal dan publik, identitas naratif
berfungsi sebagai \textbf{sistem pengarah pesan}. Ia menjawab tiga
pertanyaan fundamental komunikasi: - \emph{Mengapa} saya berbicara? -
\emph{Dari nilai apa} pesan ini lahir? - \emph{Untuk makna apa} pesan
ini disampaikan?

Tanpa identitas naratif, komunikasi berubah menjadi sekadar
performa---optimasi kata tanpa jiwa. Namun ketika komunikasi berakar
pada narasi diri, pesan memperoleh daya resonansi emosional dan etis.

Dengan demikian, komunikasi yang bermakna di era digital bukanlah soal
kecepatan, viralitas, atau kecanggihan medium, melainkan
\textbf{rekayasa makna berbasis identitas manusia}. Inilah kontribusi
konseptual yang saya ajukan dalam memahami komunikasi sebagai praktik
kemanusiaan, bukan sekadar keterampilan teknis.

\bookmarksetup{startatroot}

\chapter{UAS-3 My Innovations}\label{uas-3-my-innovations}

\section{\texorpdfstring{\textbf{Narrative Canvas: Infrastruktur
Komunikasi Berbasis Kesadaran
Diri}}{Narrative Canvas: Infrastruktur Komunikasi Berbasis Kesadaran Diri}}\label{narrative-canvas-infrastruktur-komunikasi-berbasis-kesadaran-diri}

\begin{figure}[H]

{\centering \pandocbounded{\includegraphics[keepaspectratio]{My_Innovations/../images/innovation.png}}

}

\caption{Innovation}

\end{figure}%

Inovasi yang saya ajukan lahir dari satu kesenjangan utama dalam praktik
komunikasi modern: banyak individu memiliki gagasan, tetapi tidak
memiliki \textbf{wadah reflektif} untuk mengartikulasikannya secara
bermakna.

\textbf{Narrative Canvas} adalah sebuah platform konseptual yang
dirancang sebagai ruang latihan komunikasi berbasis cerita hidup.
Platform ini tidak mengajarkan ``cara berbicara yang baik'', melainkan
\textbf{cara memahami diri sebelum berbicara}.

Narrative Canvas dibangun di atas tiga pilar utama: 1. \textbf{Refleksi
Diri} -- menggali pengalaman personal sebagai sumber makna, 2.
\textbf{Struktur Pesan} -- mengubah pengalaman menjadi narasi
komunikatif, 3. \textbf{Empati Audiens} -- menguji dampak pesan terhadap
orang lain.

Melalui sistem \emph{story prompt}, pengguna dipandu untuk mengurai
pengalaman hidup menjadi nilai, konflik, dan pembelajaran. Pesan yang
dihasilkan bukan sekadar informatif, tetapi autentik dan berakar pada
identitas.

Dalam konteks KIPP, Narrative Canvas berfungsi sebagai \textbf{mesin
pembangun kapasitas komunikasi}---menghubungkan kesadaran diri, empati,
dan efektivitas pesan. Inovasi ini menempatkan komunikasi bukan sebagai
output instan, tetapi sebagai proses pemaknaan berkelanjutan.

Narrative Canvas bukan sekadar produk, melainkan \textbf{infrastruktur
kesadaran komunikasi}.

\bookmarksetup{startatroot}

\chapter{UAS-2 My Opinions}\label{uas-2-my-opinions}

\section{\texorpdfstring{\textbf{Refleksi sebagai Tindakan Etis dalam
Budaya Komunikasi
Reaktif}}{Refleksi sebagai Tindakan Etis dalam Budaya Komunikasi Reaktif}}\label{refleksi-sebagai-tindakan-etis-dalam-budaya-komunikasi-reaktif}

\begin{figure}[H]

{\centering \pandocbounded{\includegraphics[keepaspectratio]{My_Opinions/../images/reflection.png}}

}

\caption{Reflection}

\end{figure}%

Opini saya berangkat dari pengamatan bahwa dunia digital telah menggeser
komunikasi dari proses reflektif menjadi respons impulsif. Kita hidup
dalam ekosistem yang menghargai kecepatan reaksi lebih tinggi daripada
kedalaman pemikiran. Dalam kondisi ini, berbicara sering kali lebih
cepat daripada memahami.

Menurut saya, sikap terbaik yang harus diambil manusia modern adalah
\textbf{reflektif sebelum reaktif}. Refleksi bukan kelemahan, melainkan
bentuk tanggung jawab etis dalam berkomunikasi.

Komunikasi reaktif cenderung: - mengedepankan emosi sesaat, -
mengabaikan konteks sosial, - dan memproduksi konflik alih-alih dialog.

Sebaliknya, komunikasi reflektif menuntut keberanian untuk berhenti
sejenak---menimbang nilai, dampak, dan makna pesan sebelum disampaikan.
Dalam perspektif Komunikasi Interpersonal dan Publik, refleksi adalah
mekanisme pengendalian diri yang menjaga komunikasi tetap manusiawi.

Saya berpendapat bahwa di tengah banjir opini dan suara, tindakan paling
progresif justru adalah \textbf{kesediaan untuk berpikir sebelum
berbicara}. Refleksi menjadikan komunikasi bukan sekadar ekspresi diri,
tetapi kontribusi terhadap ruang publik yang sehat.

Dengan demikian, refleksi bukan hanya sikap personal, melainkan
\textbf{tindakan sosial} yang menentukan kualitas relasi dan
keberlanjutan dialog dalam masyarakat digital.

\bookmarksetup{startatroot}

\chapter{UAS-4 My Knowledge}\label{uas-4-my-knowledge}

\section{\texorpdfstring{\textbf{Komunikasi Bermakna sebagai Pengetahuan
Sosial}}{Komunikasi Bermakna sebagai Pengetahuan Sosial}}\label{komunikasi-bermakna-sebagai-pengetahuan-sosial}

\begin{figure}[H]

{\centering \pandocbounded{\includegraphics[keepaspectratio]{My_Knowledge/../images/knowledge.png}}

}

\caption{Knowledge}

\end{figure}%

Pengetahuan terpenting yang saya peroleh dari perkuliahan ini adalah
bahwa komunikasi bukan hanya keterampilan individual, melainkan
\textbf{praktik sosial yang membentuk realitas bersama}.

Komunikasi yang bermakna lahir dari tiga kesadaran utama: - kesadaran
akan diri, - kesadaran akan orang lain, - dan kesadaran akan dampak
pesan.

Masyarakat sering memahami komunikasi sebagai kemampuan berbicara.
Namun, pembelajaran dari KIPP menunjukkan bahwa komunikasi yang sehat
justru dimulai dari \textbf{kemampuan mendengarkan, merefleksikan, dan
memahami konteks}.

Pengetahuan ini relevan bagi masyarakat karena: - mengurangi konflik
berbasis miskomunikasi, - memperkuat dialog antarindividu, - dan
membangun relasi sosial yang berkelanjutan.

Dengan menempatkan empati dan refleksi sebagai inti komunikasi,
masyarakat dapat bergerak dari budaya reaktif menuju budaya dialogis.

Komunikasi, pada akhirnya, bukan sekadar pertukaran pesan, melainkan
\textbf{proses kolektif membangun makna hidup bersama}.

\bookmarksetup{startatroot}

\chapter{UAS-5 My Professional
Reviews}\label{uas-5-my-professional-reviews}

\section{\texorpdfstring{\textbf{My Professional
Reviews}}{My Professional Reviews}}\label{my-professional-reviews}

\begin{figure}[H]

{\centering \pandocbounded{\includegraphics[keepaspectratio]{My_Professional_Reviews/../images/review.png}}

}

\caption{Review}

\end{figure}%

Profesionalisme dalam bidang komunikasi dan teknologi tidak hanya diukur
dari kemampuan mencipta, tetapi juga dari kemampuan \textbf{memberikan
kritik yang konstruktif dan menerima evaluasi secara terbuka}. Bagian
ini mendokumentasikan praktik penilaian profesional saya terhadap pesan
dan solusi yang dikembangkan dalam konteks era AI.

UAS-5 dirancang sebagai latihan reflektif untuk membangun objektivitas,
etika akademik, dan tanggung jawab profesional dalam menilai karya
publik.

\begin{center}\rule{0.5\linewidth}{0.5pt}\end{center}

\section{\texorpdfstring{\textbf{1. Professional Self-Reflection (Self
Assessment -- Menggunakan
AI)}}{1. Professional Self-Reflection (Self Assessment -- Menggunakan AI)}}\label{professional-self-reflection-self-assessment-menggunakan-ai}

Sebagai bagian dari refleksi profesional, saya melakukan
\textbf{self-assessment} terhadap pesan publik dan inovasi yang saya
kembangkan dengan menggunakan bantuan AI, berdasarkan rubrik resmi UAS.

Penilaian ini mencakup aspek: - kelayakan konseptual dan teknis, -
konsistensi nilai dan etika AI, - kejelasan komunikasi publik, - serta
potensi dampak sosial dari solusi yang diajukan.

AI digunakan sebagai \textbf{alat bantu refleksi objektif}, bukan
sebagai pengganti tanggung jawab akademik. Hasil self-assessment ini
membantu saya memastikan bahwa karya yang saya hasilkan bukan sekadar
ide, melainkan solusi yang \textbf{terukur, dapat dipertanggungjawabkan,
dan komunikatif secara profesional}.

\begin{center}\rule{0.5\linewidth}{0.5pt}\end{center}

\section{\texorpdfstring{\textbf{2. Professional Peer Review (Tanpa
AI)}}{2. Professional Peer Review (Tanpa AI)}}\label{professional-peer-review-tanpa-ai}

Saya juga melakukan \textbf{peer assessment secara manual tanpa bantuan
AI}, sesuai ketentuan UAS, terhadap karya (Masterpiece) rekan mahasiswa
lain yang URL-nya tercantum dalam daftar resmi (sama dengan saat UTS).

Dalam proses ini, saya berperan sebagai \textbf{reviewer profesional}
yang menilai karya rekan berdasarkan kriteria berikut:

\begin{itemize}
\item
  \textbf{Originalitas}\\
  Sejauh mana inovasi yang dikembangkan menghadirkan kebaruan dan tidak
  sekadar mengulang solusi yang sudah ada.
\item
  \textbf{Logika Sistem dan Konsep}\\
  Apakah arsitektur solusi yang dirancang mampu menjawab permasalahan
  secara rasional dan terstruktur.
\item
  \textbf{Kualitas Komunikasi Publik}\\
  Sejauh mana ide profesional tersebut disampaikan secara jelas,
  koheren, dan dapat dipahami oleh audiens non-spesialis.
\end{itemize}

Penilaian dilakukan dengan pendekatan etis, objektif, dan berbasis
rubrik, sehingga masukan yang diberikan bersifat membangun dan relevan
bagi pengembangan karya.

\begin{center}\rule{0.5\linewidth}{0.5pt}\end{center}

\section{\texorpdfstring{\textbf{3. Lembar Skor Penilaian Akhir
(Excel)}}{3. Lembar Skor Penilaian Akhir (Excel)}}\label{lembar-skor-penilaian-akhir-excel}

Seluruh hasil \textbf{Self-Assessment} dan \textbf{Peer Assessment}
dicatat dalam \textbf{satu file Excel Lembar Skor}, dengan ketentuan:

\begin{itemize}
\tightlist
\item
  Satu baris untuk setiap asesmen\\
\item
  Urutan baris:

  \begin{enumerate}
  \def\labelenumi{\arabic{enumi}.}
  \tightlist
  \item
    Self Assessment (menggunakan AI)
  \item
    Peer Assessment 1 (tanpa AI)
  \item
    Peer Assessment 2 (tanpa AI)
  \item
    Peer Assessment 3 (tanpa AI -- bonus)
  \end{enumerate}
\item
  File disimpan di repositori pribadi pada folder \texttt{UAS-5}
\item
  File di-\emph{link} ke portal UAS dan dilaporkan melalui MS Form UAS-1
  (TBD)
\end{itemize}

\subsection{🔗 Tautan Lembar Skor UAS}\label{tautan-lembar-skor-uas}

\begin{quote}
\textbf{Important}\\
📥 \textbf{\href{LINK-EXCEL-ANDA-DI-SINI}{Download Lembar Skor UAS
(Excel)}}
\end{quote}

\emph{(Tautan akan diperbarui dengan file resmi saat UAS berlangsung)}

\begin{center}\rule{0.5\linewidth}{0.5pt}\end{center}

\section{\texorpdfstring{\textbf{4. Status dan Validitas
Data}}{4. Status dan Validitas Data}}\label{status-dan-validitas-data}

\begin{quote}
⚠️ \textbf{Catatan Penting}\\
Bahan Peer Assessment yang valid \textbf{baru tersedia saat UAS}.\\
Oleh karena itu, pengisian skor dan tautan akhir akan disesuaikan dengan
materi resmi yang tersedia pada waktu UAS.
\end{quote}

\begin{center}\rule{0.5\linewidth}{0.5pt}\end{center}

\section{\texorpdfstring{\textbf{Refleksi
Profesional}}{Refleksi Profesional}}\label{refleksi-profesional}

Penilaian ini merepresentasikan upaya saya untuk bersikap objektif,
etis, dan reflektif dalam menilai kualitas pekerjaan---baik karya
sendiri maupun karya orang lain---di lingkungan profesional Sistem dan
Teknologi Informasi.

Kemampuan melakukan evaluasi yang adil dan komunikatif merupakan bagian
penting dari \textbf{kompetensi profesional di era AI}.

\begin{center}\rule{0.5\linewidth}{0.5pt}\end{center}

\bookmarksetup{startatroot}

\chapter{Summary}\label{summary}

In summary, this book has no content whatsoever.

\bookmarksetup{startatroot}

\chapter*{References}\label{references}
\addcontentsline{toc}{chapter}{References}

\markboth{References}{References}

\phantomsection\label{refs}




\end{document}
