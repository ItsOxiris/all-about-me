% Options for packages loaded elsewhere
% Options for packages loaded elsewhere
\PassOptionsToPackage{unicode}{hyperref}
\PassOptionsToPackage{hyphens}{url}
\PassOptionsToPackage{dvipsnames,svgnames,x11names}{xcolor}
%
\documentclass[
  letterpaper,
  DIV=11,
  numbers=noendperiod]{scrreprt}
\usepackage{xcolor}
\usepackage{amsmath,amssymb}
\setcounter{secnumdepth}{5}
\usepackage{iftex}
\ifPDFTeX
  \usepackage[T1]{fontenc}
  \usepackage[utf8]{inputenc}
  \usepackage{textcomp} % provide euro and other symbols
\else % if luatex or xetex
  \usepackage{unicode-math} % this also loads fontspec
  \defaultfontfeatures{Scale=MatchLowercase}
  \defaultfontfeatures[\rmfamily]{Ligatures=TeX,Scale=1}
\fi
\usepackage{lmodern}
\ifPDFTeX\else
  % xetex/luatex font selection
\fi
% Use upquote if available, for straight quotes in verbatim environments
\IfFileExists{upquote.sty}{\usepackage{upquote}}{}
\IfFileExists{microtype.sty}{% use microtype if available
  \usepackage[]{microtype}
  \UseMicrotypeSet[protrusion]{basicmath} % disable protrusion for tt fonts
}{}
\makeatletter
\@ifundefined{KOMAClassName}{% if non-KOMA class
  \IfFileExists{parskip.sty}{%
    \usepackage{parskip}
  }{% else
    \setlength{\parindent}{0pt}
    \setlength{\parskip}{6pt plus 2pt minus 1pt}}
}{% if KOMA class
  \KOMAoptions{parskip=half}}
\makeatother
% Make \paragraph and \subparagraph free-standing
\makeatletter
\ifx\paragraph\undefined\else
  \let\oldparagraph\paragraph
  \renewcommand{\paragraph}{
    \@ifstar
      \xxxParagraphStar
      \xxxParagraphNoStar
  }
  \newcommand{\xxxParagraphStar}[1]{\oldparagraph*{#1}\mbox{}}
  \newcommand{\xxxParagraphNoStar}[1]{\oldparagraph{#1}\mbox{}}
\fi
\ifx\subparagraph\undefined\else
  \let\oldsubparagraph\subparagraph
  \renewcommand{\subparagraph}{
    \@ifstar
      \xxxSubParagraphStar
      \xxxSubParagraphNoStar
  }
  \newcommand{\xxxSubParagraphStar}[1]{\oldsubparagraph*{#1}\mbox{}}
  \newcommand{\xxxSubParagraphNoStar}[1]{\oldsubparagraph{#1}\mbox{}}
\fi
\makeatother


\usepackage{longtable,booktabs,array}
\usepackage{calc} % for calculating minipage widths
% Correct order of tables after \paragraph or \subparagraph
\usepackage{etoolbox}
\makeatletter
\patchcmd\longtable{\par}{\if@noskipsec\mbox{}\fi\par}{}{}
\makeatother
% Allow footnotes in longtable head/foot
\IfFileExists{footnotehyper.sty}{\usepackage{footnotehyper}}{\usepackage{footnote}}
\makesavenoteenv{longtable}
\usepackage{graphicx}
\makeatletter
\newsavebox\pandoc@box
\newcommand*\pandocbounded[1]{% scales image to fit in text height/width
  \sbox\pandoc@box{#1}%
  \Gscale@div\@tempa{\textheight}{\dimexpr\ht\pandoc@box+\dp\pandoc@box\relax}%
  \Gscale@div\@tempb{\linewidth}{\wd\pandoc@box}%
  \ifdim\@tempb\p@<\@tempa\p@\let\@tempa\@tempb\fi% select the smaller of both
  \ifdim\@tempa\p@<\p@\scalebox{\@tempa}{\usebox\pandoc@box}%
  \else\usebox{\pandoc@box}%
  \fi%
}
% Set default figure placement to htbp
\def\fps@figure{htbp}
\makeatother





\setlength{\emergencystretch}{3em} % prevent overfull lines

\providecommand{\tightlist}{%
  \setlength{\itemsep}{0pt}\setlength{\parskip}{0pt}}



 


\KOMAoption{captions}{tableheading}
\makeatletter
\@ifpackageloaded{bookmark}{}{\usepackage{bookmark}}
\makeatother
\makeatletter
\@ifpackageloaded{caption}{}{\usepackage{caption}}
\AtBeginDocument{%
\ifdefined\contentsname
  \renewcommand*\contentsname{Table of contents}
\else
  \newcommand\contentsname{Table of contents}
\fi
\ifdefined\listfigurename
  \renewcommand*\listfigurename{List of Figures}
\else
  \newcommand\listfigurename{List of Figures}
\fi
\ifdefined\listtablename
  \renewcommand*\listtablename{List of Tables}
\else
  \newcommand\listtablename{List of Tables}
\fi
\ifdefined\figurename
  \renewcommand*\figurename{Figure}
\else
  \newcommand\figurename{Figure}
\fi
\ifdefined\tablename
  \renewcommand*\tablename{Table}
\else
  \newcommand\tablename{Table}
\fi
}
\@ifpackageloaded{float}{}{\usepackage{float}}
\floatstyle{ruled}
\@ifundefined{c@chapter}{\newfloat{codelisting}{h}{lop}}{\newfloat{codelisting}{h}{lop}[chapter]}
\floatname{codelisting}{Listing}
\newcommand*\listoflistings{\listof{codelisting}{List of Listings}}
\makeatother
\makeatletter
\makeatother
\makeatletter
\@ifpackageloaded{caption}{}{\usepackage{caption}}
\@ifpackageloaded{subcaption}{}{\usepackage{subcaption}}
\makeatother
\usepackage{bookmark}
\IfFileExists{xurl.sty}{\usepackage{xurl}}{} % add URL line breaks if available
\urlstyle{same}
\hypersetup{
  pdftitle={Armein Z. R. Langi},
  pdfauthor={131902360 Armein Z R Langi},
  colorlinks=true,
  linkcolor={blue},
  filecolor={Maroon},
  citecolor={Blue},
  urlcolor={Blue},
  pdfcreator={LaTeX via pandoc}}


\title{Armein Z. R. Langi}
\usepackage{etoolbox}
\makeatletter
\providecommand{\subtitle}[1]{% add subtitle to \maketitle
  \apptocmd{\@title}{\par {\large #1 \par}}{}{}
}
\makeatother
\subtitle{Portfolio Asesmen II-2100 KIPP}
\author{131902360 Armein Z R Langi}
\date{2025-09-15}
\begin{document}
\maketitle

\renewcommand*\contentsname{Table of contents}
{
\hypersetup{linkcolor=}
\setcounter{tocdepth}{2}
\tableofcontents
}

\bookmarksetup{startatroot}

\chapter*{Selamat Berjumpa}\label{selamat-berjumpa}
\addcontentsline{toc}{chapter}{Selamat Berjumpa}

\markboth{Selamat Berjumpa}{Selamat Berjumpa}

\begin{figure}[H]

{\centering \includegraphics[width=9.5\linewidth,height=\textheight,keepaspectratio]{images/AZRL.png}

}

\caption{About Me}

\end{figure}%

Armein Z R Langi adalah Guru Besar di Sekolah Teknik Elektro dan
Informatika ITB, dosen ITB sejak Desember 1987, mantan Rektor
Universitas Kristen Maranatha, 1 Maret 2016 s/d 29 Februari 2020, mantan
Kepala Pusat Penelitian Teknologi Informasi dan Komunikasi (PP-TIK) ITB
November 2005 s/d Maret 2010, dan Sekretaris MWA ITB Mei 2010-Jan 2011.

Lahir di Tomohon 1962 dari pasangan Manado dan Sunda. Saat ini tinggal
di Bandung, menikah dengan Ina dan dikaruniai empat anak. Ayah dari
Gladys, Kezia, Andria, dan Marco.

Sharing pikiran singkat ada di blog \url{https://azrl.wordpress.com}.
Facebookk: armein\_langi

Andria, putri saya, lumayan sering membaca blog ini. Dia juga sumber
inspirasi tulisan saya. Dia senang menceritakan jokes pada saya, dan
kalau bagus saya tulis di sini, terutama seri Kocak Ala Andria. Dan dia
suka heran, karena isinya sudah berbeda. Memang saya senang
mengubah-ngubah cerita karena saya tidak suka menjiplak mentah-mentah.
Minggu lalu tidak sengaja dia memuji tulisan blog ini.

``Papa\ldots{} papa\ldots{} ``, tanyanya, ``Tulisan blog papa itu
copy-paste dari tulisan orang ya\ldots?''

Haa? Ya nggak mungkin lah. Kecuali lirik lagu, semua artikel di sini
ditulis sendiri.

``Ah bohong, soalnya tulisannya terlalu banyak\ldots.'' katanya tetap
tidak percaya, ``Nggak mungkinlah papa tulis sendiri\ldots{}''

Hehe, saya tersenyum sambil membelai Andria, karena buat saya itu adalah
ultimate compliment. Thanks sweetheart\ldots{}

Jadi kalau orang tidak percaya bahwa sesuatu itu karya anda, anda yang
buat, jangan marah. Itu adalah pujian yang sejati.

\bookmarksetup{startatroot}

\chapter{UTS-1 All About Me}\label{uts-1-all-about-me}

Naufalziyadh Alif Bintang Shaeky (Falzi) 18224077 --- Sistem dan
Teknologi Informasi Halo semuanya, aku Falzi. Katanya sih aku ini ESFP,
si penghibur yang energinya kayak gak habis-habis. Buat aku, hidup tuh
kayak panggung besar, dan tiap hari selalu ada aja hal baru yang bisa
dibikin seru. Dunia ini gak cuma tentang masalah, tapi tentang gimana
kita nikmatin prosesnya bareng orang-orang di sekitar kita. Kuliah di
STI ngajarin aku soal logika dan sistem, tapi jujur, tempat ternyaman
aku justru di tengah keramaian. Aku paling senang ikut kegiatan, ngobrol
sama orang baru, dan ngerasain energi dari suasana yang rame. Buatku,
hal paling keren dari sebuah sistem adalah interaksi antar manusia.
Soalnya, suasana asik itu muncul kalau semua orang bisa nyatu dan
ngerasa senang bareng. Aku juga termasuk orang yang peka sama suasana.
Kalau lagi kumpul terus suasananya mulai kaku, aku sering jadi yang
pertama sadar. Biasanya langsung mikir, ``kok sepi sih?'' Terus refleks
nyari cara biar semua orang bisa ketawa lagi. Kadang cukup lewat obrolan
kecil atau dengerin temen curhat. Yang penting semua orang ngerasa
nyaman. Beberapa waktu terakhir aku sempat ngalamin beberapa hal yang
gak enak, termasuk soal akademik. Tapi aku belajar buat gak terus
tenggelam di situ. Aku lihat itu sebagai kesempatan buat tumbuh dan
belajar hal baru. Kalau satu jalan ketutup, ya cari cara lain. Kadang
malah ketemu hal yang lebih seru dari yang direncanain. Aku gak selalu
semangat setiap waktu, kadang juga turun. Tapi justru dari situ aku
belajar buat bangun hubungan yang kuat sama orang lain. Hubungan yang
gak cuma karena seru di awal, tapi juga karena kita saling dukung dan
peduli. Mimpiku sederhana. Aku pengen punya lingkungan pertemanan yang
jadi tempatnya belajar bareng, saling peduli, dan bebas nyalurin ide.
Aku pengen dikelilingi orang-orang yang bukan cuma pintar, tapi juga
punya hati yang hangat. Kalau boleh dibilang, aku nyari temen yang jago
mikir tapi juga baik hati. Temen yang bisa ngobrol dari hal receh sampai
hal berat, tapi tetap bikin semua orang ngerasa diterima. Tempat di mana
kita ngerasa punya rumah bareng.

Itu aku, Falzi. Anak ESFP yang suka bikin suasana rame, tapi juga senang
denger cerita orang. Buat aku, hidup itu soal berbagi energi positif dan
bikin setiap momen jadi sesuatu yang bisa diinget bareng-bareng. Salam
kenal ya, halo semua!

\bookmarksetup{startatroot}

\chapter{}\label{section}

(Verse 1) Hei, kamu! Ngapain termangu? Mata menatap jauh ke sepatu
Mikirin omongan si itu, si ini Takut salah, takut nggak keren lagi

Dunia berputar cepat sekali Tik-tok, nggak akan menanti Kamu masih di
situ, sibuk meragu Kapan mau maju?

(Pre-Chorus) Jangan cuma jadi penonton Yang tepuk tangan sambil gigit
jempol Hidup ini bukan film orang Ini panggungmu, ayo dong!

(Chorus) Hidup ini cuma sekali! (Hei!) Coba semua yang kau ingini! Jatuh
bangun itu pasti! Jangan sampai menyesal nanti! Aw aw aw awww Mendingan
gagal daripada tak nyali! Aw aw aw awww Gas aja dulu, pikir belakangan!

(Verse 2) Mau cat rambut warna pelangi? Mau teriak di puncak tertinggi?
Mau belajar dansa atau main biola? Kenapa harus nunggu tua?

Standar mereka bikin gila Nggak ada habisnya, percuma Ini badanmu, ini
suaramu Terserah kamu, mau dibawa ke mana!

(Pre-Chorus) Jangan cuma jadi penonton Yang tepuk tangan sambil gigit
jempol Hidup ini bukan film orang Ini panggungmu, ayo dong!

(Chorus) Hidup ini cuma sekali! (Hei!) Coba semua yang kau ingini! Jatuh
bangun itu pasti! Jangan sampai menyesal nanti! Aw aw aw awww Mendingan
gagal daripada tak nyali! Aw aw aw awww Gas aja dulu, pikir belakangan!

(Bridge) Mungkin nanti kamu lebam Mungkin ceritanya kelam Tapi itu jauh
lebih baik Daripada buku kosong tanpa coretan!

Aku cuma mau bilang\ldots{} Dengerin aku\ldots{} SAYANGGGGGGGG!!! Ayo
bangun!

(Chorus) Hidup ini cuma sekali! (Hei!) Coba semua yang kau ingini! Jatuh
bangun itu pasti! Jangan sampai menyesal nanti! Aw aw aw awww Mendingan
gagal daripada tak nyali! Aw aw aw awww Gas aja dulu, pikir belakangan!

(Outro) Coba semuanya! (Jangan menyesal!) Hidup sekali! (Coba semuanya!)
Aw aw aw awww! Ya! Jangan sampai menyesal! (Musik berhenti mendadak di
ketukan terakhir)

\bookmarksetup{startatroot}

\chapter{UTS-3 My Stories for You}\label{uts-3-my-stories-for-you}

My Story For You Halo, ini aku lagi belajar buat UTS Jarkom, tapi
tiba-tiba keinget kalau ada tugas bikin cerita. Jadi yaudah, cerita dulu
deh sebelum otakku keburu meledak gara-gara subnetting dan protokol yang
entah kenapa nggak masuk-masuk ke kepala. Kadang, di tengah tumpukan
catatan dan angka-angka aneh itu, aku suka kepikiran hal-hal yang justru
nggak ada hubungannya sama pelajaran. Kayak hari ini, aku kepikiran
semua perjalanan yang udah kulewatin sejak pertama kali masuk ITB.
Mungkin ini waktu yang pas buat berhenti sebentar, tarik napas, dan
ngeliat ke belakang, bukan buat nyesel, tapi buat nginget gimana
semuanya dimulai. Kalau ditarik mundur ke akhir tahun 2023, hidupku
waktu itu cuma seputar tryout dan pengumuman. Masa-masa SNBP adalah fase
di mana aku ngerasa semuanya bisa dikontrol. Aku yakin banget bisa masuk
universitas impian tanpa drama. Tapi ternyata, dunia punya cara unik
buat bikin kita sadar bahwa nggak semua bisa sesuai rencana. Tryout
pertamaku cuma dapat nilai lima ratusan, angka yang awalnya bikin aku
bengong lama di depan layar. Dari situ aku baru ngerti, percaya diri
tanpa usaha itu sama aja kayak jalan di kabut, keliatannya aman, padahal
nyasar pelan-pelan. Tapi nilai kecil itu justru jadi titik balik. Aku
mulai rajin belajar, ikut bimbingan, dan pelan-pelan mulai ngerasa bahwa
mungkin kegagalan itu bukan akhir, tapi peringatan kecil supaya aku
nggak main-main. Hari pengumuman SNBP datang, dan seperti dugaan buruk
yang berulang di kepala, aku gagal. Aku nggak diterima. Rasanya waktu
itu kayak seluruh usaha selama berbulan-bulan tiba-tiba nggak ada
artinya. Tapi anehnya, setelah nangis bentar dan diem lama, aku malah
ngerasa tenang. Aku bilang ke diri sendiri, mungkin Tuhan cuma nyuruh
aku muter dikit sebelum nyampe. Lalu aku ikut SNBT. Hasilnya? Nyaris.
Tinggal satu langkah lagi, tapi kursi itu bukan buatku. Kalau dipikir
sekarang, mungkin itu cara semesta bilang, ``Belum, tapi nanti.'' Jadi
aku lanjut ambil ujian mandiri ITB. Aku belajar siang malam, bahkan
sehari setelah wisuda masih buka catatan. Aku inget banget, ustaz di
pondok ngasih doa panjang banget sebelum aku berangkat. Dan ternyata,
doa itu beneran nyampe. Aku diterima di STEI-K ITB. Waktu baca
pengumuman itu, aku cuma bisa bengong. Dunia yang sempet terasa gelap
mendadak terang, dan aku nangis, bukan karena sedih, tapi karena
akhirnya aku bisa lega. Masuk ITB, rasanya kayak dilempar ke dunia baru
yang seru sekaligus menakutkan. Hari pertama kuliah, kampus lain sibuk
ospek, tapi kami malah dikasih mata kuliah Pancasila. Lucu banget,
pikirku waktu itu. Aku duduk di kelas, nggak kenal siapa-siapa, bingung
mau mulai dari mana. Tapi dari kelas itulah aku pertama kali ngerasa
punya tempat. Aku sekelompok sama orang-orang yang ternyata asik banget.
Dari kerja kelompok, ngobrolin hal random, sampai akhirnya sering
nongkrong bareng. Mereka yang awalnya cuma nama di daftar hadir,
lama-lama jadi orang yang selalu aku cari tiap kali ada kelas baru. Di
situlah aku sadar, ternyata pertemanan bisa tumbuh dari hal sekecil
``ayo kerjain tugas bareng, yuk.'' Lalu datang OSKM. Ah, ini bagian yang
susah banget dilupain. Capek, rame, berisik, tapi jujur, aku kangen.
Hari-harinya penuh tawa, teriakan, dan sedikit drama. Aku ketemu banyak
orang baru, bahkan sempet suka sama seseorang meski cuma sebentar. Tapi
justru dari situ aku belajar hal yang jauh lebih penting. OSKM ngajarin
aku cara ngeliat orang bukan cuma dari luar. Di antara semua keseruan
dan kehebohan itu, aku mulai ngerti gimana tiap orang punya ceritanya
sendiri. Ada yang berjuang, ada yang sembunyi di balik senyum, ada juga
yang cuma pengen didengar. Dan di tengah kerumunan itu, aku ngerasa
hidup. Setelah OSKM, hidup di kampus mulai berjalan seperti biasa.
Sampai akhirnya aku ikut kepanitiaan pertamaku: Sekolah Tour. Awalnya
cuma karena temen ngajak, aku mikirnya bakal seru aja kalau rame-rame.
Tapi ternyata dari keputusan se-random itu, banyak banget hal yang
berubah. Di awal, aku sempat nggak aktif beberapa hari karena sibuk sama
urusan lain. Waktu balik lagi, suasananya udah beda. Semua orang udah
akrab, udah punya ritme sendiri, dan aku merasa asing di antara tawa
mereka. Tapi orang-orang di sana ternyata baik banget. Mereka tetap
nyambut aku kayak nggak pernah ketinggalan apa-apa. Pelan-pelan aku
mulai nyaman lagi. Hari itu, kami keliling lab rumpun elektronika.
Mentor kami luar biasa enerjik. Suaranya lantang, matanya nyala waktu
ngejelasin setiap alat di meja. Aku cuma bisa manggut-manggut, berusaha
kelihatan paham padahal dalam hati cuma, ``Ini apaan sih?'' Tapi di
balik semua itu, suasananya hangat banget. Kami foto bareng, ketawa
bareng, dan bahkan sempat bercanda sampai lupa waktu. Tapi ada satu
momen di mana aku tiba-tiba merasa diam. Di tengah tawa orang-orang, aku
ngerasa jauh, kayak cuma penonton di cerita yang harusnya aku mainin
sendiri. Mungkin itu perasaan capek, tapi entah kenapa, bagian itu
justru paling kuingat sampai sekarang. Begitu acara selesai, aku pulang
ke Bandung. Rumahku kecil, tapi selalu terasa besar setiap kali aku
masuk. Bunda lagi masak di dapur, wangi tumisan langsung nyambut dari
pintu. ``Gimana hari ini?'' tanya Bunda. Aku cuma bisa nyengir, ``Capek
banget, Bun. Rasanya pengen tidur seharian.'' Tapi Bunda cuma ketawa
kecil dan bilang, ``Capek itu tandanya kamu lagi ngelakuin sesuatu yang
berharga.'' Aku diem lama waktu itu. Kalimat sederhana yang entah kenapa
terasa dalam banget. Mungkin itu alasan kenapa setiap kali aku ngerasa
lelah, aku selalu inget perkataan itu. Setelah hari itu, semuanya
berjalan cepat. UAS datang, tugas menumpuk, dan kepanitiaan mulai terasa
jauh. Grup panitia di HP cuma kubaca sekilas, niat bales tapi
ujung-ujungnya malah aku tutup lagi. Kadang aku pengen aktif lagi, tapi
selalu ada suara kecil di kepala yang bilang, ``Udah telat.'' Sampai
suatu malam, ada pesan masuk dari salah satu panitia. Pesannya pendek
aja, nanyain apakah aku masih mau ikut kegiatan lanjutan. Aku sempat
ragu mau jawab apa. Tapi kalimat terakhirnya bikin aku diem lama,
``Kalau gitu kamu ikut UAS-nya kan, kann?'' Entah kenapa, kalimat
sesederhana itu bikin dadaku anget. Ada orang yang masih inget aku,
bahkan ketika aku sendiri udah mulai lupa. Dan dari situ aku ngerti,
kadang yang kita butuhin cuma satu orang yang ngulurin tangan duluan.
Itu malam aku mikir lama. Tentang betapa seringnya aku mundur cuma
karena takut ketinggalan. Padahal sebenarnya nggak ada yang benar-benar
terlambat kalau kita mau balik lagi. Mungkin makna ``ikatan'' bukan cuma
soal siapa yang selalu bareng, tapi tentang siapa yang masih inget kita
ketika kita nggak ada. Sekolah Tour akhirnya bukan cuma kegiatan kampus
buatku, tapi titik di mana aku belajar tentang hubungan. Tentang rasa
lelah yang ternyata nggak sia-sia, tentang teman yang nggak pernah
berhenti percaya, dan tentang diriku sendiri yang belajar buat nggak
takut mulai dari awal. Sekarang, di tengah malam yang dingin dan
tumpukan catatan Jarkom yang belum aku sentuh lagi, aku senyum sendiri.
Dari anak SMA yang dulu gagal tryout, sekarang aku duduk di kampus
impian, nulis cerita tentang perjalanan yang nggak sempurna tapi nyata.
Semua perjuangan, tawa, bahkan rasa ragu, ternyata punya tempatnya
masing-masing. Mungkin nanti kalau aku baca tulisan ini lagi, aku bakal
bilang ke diri sendiri, ``Lihat, kamu dulu cuma pengen lulus tryout,
tapi sekarang kamu lagi nulis cerita hidupmu sendiri.'' Dan mungkin di
situlah inti dari semuanya. Bukan soal seberapa cepat sampai, tapi
seberapa tulus kita jalanin setiap langkahnya, bareng orang-orang yang
selalu ngingetin kita buat terus maju, meski pelan.

\bookmarksetup{startatroot}

\chapter{UTS-4 My SHAPE (Spiritual Gifts, Heart, Abilities, Personality,
Experiences)}\label{uts-4-my-shape-spiritual-gifts-heart-abilities-personality-experiences}

My SHAPE Halo, aku Falzi. Kali ini aku mau cerita tentang satu hal yang
lumayan menarik buat aku, yaitu My SHAPE. Buatku, ini bukan sekadar
tugas refleksi, tapi cara buat lebih kenal sama diri sendiri. Aku pengen
jujur dan apa adanya, karena setiap orang punya caranya sendiri buat
tumbuh dan bersinar. S --- Signature Strengths (Kekuatan Utamaku) Kalau
ditanya apa kekuatanku, aku nggak langsung mikir soal kemampuan
akademik. Aku lebih ngerasa kekuatanku ada di energi positif dan caraku
berinteraksi sama orang lain. Aku paling hidup kalau bisa ngobrol, kerja
bareng, atau bikin suasana jadi lebih santai. Dari situ biasanya muncul
ide-ide spontan yang nggak kepikiran sebelumnya. Selain itu, aku cepat
banget menyesuaikan diri. Di tempat baru, aku nggak butuh waktu lama
buat nyatu sama lingkungan. Aku senang denger cerita orang, jadi aku
mudah nyambung dengan berbagai karakter. Rasanya kayak punya tombol
``connect'' yang langsung aktif setiap kali ketemu orang baru. H ---
Heart (Hal yang Bikin Aku Peduli dan Bersemangat) Bagian ini yang paling
ngena buat aku. Hatiku paling hidup kalau lagi di tengah-tengah orang
yang punya semangat yang sama. Aku suka suasana ramai yang hangat, di
mana semua orang bisa jadi diri sendiri. Dari situ aku ngerasa, ternyata
kebahagiaan itu bukan soal pencapaian pribadi, tapi soal kebersamaan.
Aku punya nilai yang selalu aku pegang, yaitu membuat lingkungan di
sekitarku jadi lebih menyenangkan. Aku suka jadi bagian dari sesuatu
yang bisa bikin orang lain ngerasa diterima dan dihargai. Karena aku
percaya, kebahagiaan itu menular. A --- Aptitudes \& Acquired Skills
(Bakat dan Kemampuan yang Aku Kembangin) Kalau dibayangin kayak
perjalanan game, aku punya dua jenis kemampuan. Pertama, kemampuan alami
yang udah ada dari dulu, dan kedua, kemampuan yang aku bangun selama
proses belajar di kampus. Kemampuan alami yang paling aku rasain adalah
mudah bergaul dan punya empati. Aku bisa ngerasain perubahan suasana dan
tahu kapan harus ngomong, kapan harus dengerin. Ini bantu banget waktu
kerja kelompok atau ngurus kegiatan bareng teman-teman. Kemampuan yang
aku kembangin sekarang lebih ke arah analisis dan pemecahan masalah. Di
STI, aku belajar cara berpikir sistematis dan mencari solusi yang
efisien. Tapi aku juga pengen kemampuan itu tetap punya sentuhan
manusiawi, supaya hasilnya nggak cuma berguna secara teknis tapi juga
bisa membantu kehidupan orang lain. Selain itu, aku juga terus ngasah
kemampuan bicara di depan umum, karena aku suka banget berbagi energi
positif ke orang banyak. P --- Personality (Kepribadian yang Ngebentuk
Cara Aku Melihat Dunia) Aku termasuk orang yang spontan, terbuka, dan
penuh rasa ingin tahu. Aku lebih suka langsung turun ke lapangan
daripada cuma mikirin teori. Hal baru selalu bikin aku semangat, apalagi
kalau bisa ngerasain langsung pengalaman itu. Aku juga tipe orang yang
fokus ke momen sekarang. Aku percaya hidup itu lebih bermakna kalau
dijalani sepenuh hati, bukan cuma dipikirin. Kadang aku bisa terlalu
terbawa suasana, tapi dari situ juga aku belajar banyak hal tentang
keseimbangan antara perasaan dan logika. E --- Experiences (Pengalaman
yang Ngebentuk Aku Sampai Sekarang) Setiap pengalaman yang aku lewatin
punya ceritanya sendiri. Ada yang lucu, ada yang bikin jatuh, tapi
semuanya berharga. Aku pernah ngalamin masa-masa di mana semangatku
turun karena hasil yang nggak sesuai harapan. Tapi dari situ aku belajar
buat nggak berhenti. Aku sadar kalau tumbuh itu nggak selalu terasa
enak, tapi justru dari situ kita tahu sejauh apa kita udah berjalan.
Setiap kegagalan ngasih pelajaran tentang sabar dan percaya proses. Dan
setiap pertemuan dengan orang baru selalu ngasih sudut pandang yang
bikin aku berkembang. Semua itu ngebentuk aku jadi orang yang lebih kuat
dan lebih peduli. Itu versi SHAPE-ku. Buatku, ini bukan cuma kumpulan
poin tentang diri sendiri, tapi juga perjalanan buat mengenal siapa aku
sebenarnya. Aku belajar bahwa kekuatan itu bukan selalu soal kemampuan
besar, tapi soal keinginan buat terus mencoba, peduli, dan belajar dari
setiap momen. Hidup mungkin bisa dibilang kayak panggung, tapi aku nggak
pengen cuma tampil di atasnya. Aku pengen bikin panggung itu jadi tempat
di mana semua orang bisa bersinar bareng, saling dukung, dan tumbuh
sama-sama. Karena di akhir hari, yang paling penting bukan seberapa
besar cahayamu, tapi seberapa banyak cahaya yang bisa kamu bagikan ke
sekitar.

\bookmarksetup{startatroot}

\chapter*{Selamat Berjumpa}\label{selamat-berjumpa-1}
\addcontentsline{toc}{chapter}{Selamat Berjumpa}

\markboth{Selamat Berjumpa}{Selamat Berjumpa}

\begin{figure}[H]

{\centering \includegraphics[width=9.5\linewidth,height=\textheight,keepaspectratio]{My_Personal_Reviews/images/AZRL.png}

}

\caption{About Me}

\end{figure}%

Armein Z R Langi adalah Guru Besar di Sekolah Teknik Elektro dan
Informatika ITB, dosen ITB sejak Desember 1987, mantan Rektor
Universitas Kristen Maranatha, 1 Maret 2016 s/d 29 Februari 2020, mantan
Kepala Pusat Penelitian Teknologi Informasi dan Komunikasi (PP-TIK) ITB
November 2005 s/d Maret 2010, dan Sekretaris MWA ITB Mei 2010-Jan 2011.

Lahir di Tomohon 1962 dari pasangan Manado dan Sunda. Saat ini tinggal
di Bandung, menikah dengan Ina dan dikaruniai empat anak. Ayah dari
Gladys, Kezia, Andria, dan Marco.

Sharing pikiran singkat ada di blog \url{https://azrl.wordpress.com}.
Facebookk: armein\_langi

Andria, putri saya, lumayan sering membaca blog ini. Dia juga sumber
inspirasi tulisan saya. Dia senang menceritakan jokes pada saya, dan
kalau bagus saya tulis di sini, terutama seri Kocak Ala Andria. Dan dia
suka heran, karena isinya sudah berbeda. Memang saya senang
mengubah-ngubah cerita karena saya tidak suka menjiplak mentah-mentah.
Minggu lalu tidak sengaja dia memuji tulisan blog ini.

``Papa\ldots{} papa\ldots{} ``, tanyanya, ``Tulisan blog papa itu
copy-paste dari tulisan orang ya\ldots?''

Haa? Ya nggak mungkin lah. Kecuali lirik lagu, semua artikel di sini
ditulis sendiri.

``Ah bohong, soalnya tulisannya terlalu banyak\ldots.'' katanya tetap
tidak percaya, ``Nggak mungkinlah papa tulis sendiri\ldots{}''

Hehe, saya tersenyum sambil membelai Andria, karena buat saya itu adalah
ultimate compliment. Thanks sweetheart\ldots{}

Jadi kalau orang tidak percaya bahwa sesuatu itu karya anda, anda yang
buat, jangan marah. Itu adalah pujian yang sejati.

\bookmarksetup{startatroot}

\chapter{Summary}\label{summary}

In summary, this book has no content whatsoever.

\bookmarksetup{startatroot}

\chapter*{References}\label{references}
\addcontentsline{toc}{chapter}{References}

\markboth{References}{References}

\phantomsection\label{refs}




\end{document}
